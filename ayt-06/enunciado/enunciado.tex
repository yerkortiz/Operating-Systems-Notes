\documentclass{article}
\usepackage[utf8]{inputenc}
\usepackage{epigraph}
\usepackage{listings}
\usepackage[spanish]{babel}
\usepackage{color}
\usepackage{graphicx}
\usepackage{tikz}
%\usepackage{fontspec}
\usepackage[letterpaper, total={7in, 10in}]{geometry}
%\setmainfont{Neo Euler} % IBM 3270, TERMINESS TTF
\title{\centerline{\textbf{SISTEMAS OPERATIVOS: AYUDANTÍA 6}}}
\author{\centerline{Ayudante: Yerko Ortiz}}
\date{}
\definecolor{mygreen}{rgb}{0,0.6,0}
\definecolor{mygray}{rgb}{0.5,0.5,0.5}
\definecolor{mymauve}{rgb}{0.58,0,0.82}

\lstset{ %
  backgroundcolor=\color{white},   % choose the background color
  basicstyle=\footnotesize,        % size of fonts used for the code
  breaklines=true,                 % automatic line breaking only at whitespace
  captionpos=b,                    % sets the caption-position to bottom
  commentstyle=\color{mygreen},    % comment style
  escapeinside={\%*}{*)},          % if you want to add LaTeX within your code
  keywordstyle=\color{blue},       % keyword style
  stringstyle=\color{mymauve},     % string literal style
  showstringspaces=false
}
\begin{document}
\maketitle
\begin{flushleft}
\textbf{Objetivo de la ayudantía: Reforzar conceptos referentes a mecanismos de sincronización.}
\vspace{1cm}
\hrule
\vspace{1cm}
\section*{\centerline{Productor-Consumidor}}
Un proceso productor genera números enteros(pueden ser objetos o cualquier tipo de dato) y los almacena en un buffer. Existe un proceso consumidor que recolecta lo que el proceso consumidor produce, el buffer puede contener un máximo de N elementos. 
Diseñe e implemente un programa en C que permita la ejecución concurrente de ambos procesos, usando algún mecanismo de sincronización visto en clases.
\section*{\centerline{Construyendo moleculas de agua}}
Existen dos tipos de thread, los de oxígeno y los de hidrógeno. Con el fin de
usar estos threads para ensamblar moleculas de agua, es necesario crear una barrera
de forma que cada thread espera hasta que una molecula de agua esté lista para luego
continuar creando moleculas. \newline
\begin{itemize}
\item Si un oxígeno llega a la barrera, entonces debe esperar a que lleguen dos threads
de hidrógeno. 
\item Si un hidrógeno llega a la barrera, entonces este ha de esperar a que llegue un thread
    de hidrógeno y uno de oxígeno.
\end{itemize}
Diseñe e implemente un programa en C que pueda manejar la concurrencia de threads de hidrógeno y oxígeno, para crear moleculas de agua.
\centerline{\textbf{Gracias por su atención!}}
\end{flushleft}
\end{document}



